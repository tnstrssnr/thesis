\documentclass{beamer}
\usetheme[deutsch]{KIT}

\usepackage[utf8]{inputenc}
\usepackage[T1]{fontenc}
\usepackage{babel}
\usepackage{tikz,calc,ifthen}
\usepackage{mathtools}
\usepackage[normalem]{ulem}
\usepackage{graphicx}

\usetikzlibrary{positioning,calc,arrows,shapes}
\tikzset{
  every node/.style={transform shape},
  auto,
  block/.style={align=center,rectangle,draw,minimum height=20pt,minimum width=30pt},
  >=triangle 60,
  alt/.code args={<#1>#2#3}{%
      \alt<#1>{\pgfkeysalso{#2}}{\pgfkeysalso{#3}}
  },
  beameralert/.style={alt=<#1>{color=green!80!black}{}},
  mythick/.style={line width=1.4pt}
}

\newcommand*{\maxwidthofm}[2]{\maxof{\widthof{$#1$}}{\widthof{$#2$}}}
\newcommand<>*{\robustaltm}[2]{
  \alt#3
  {\mathmakebox[\maxwidthofm{#1}{#2}]{#1}\vphantom{#1#2}}
    {\mathmakebox[\maxwidthofm{#1}{#2}]{#2}\vphantom{#1#2}}
}

\newcommand<>*{\nodealert}[1]{\only#2{\draw[overlay,mythick,color=green!80!black] (#1.north west) rectangle (#1.south east)}}

\title{Evaluation of Randomized Algorithms}
\author{Alice Müller}
\subtitle{\insertauthor}
\institute[IPD]{Lehrstuhl Programmierparadigmen, IPD Snelting}
\date{29.10.2013}
\KITtitleimage{cover.png}

\begin{document}

\begin{frame}
  \maketitle
\end{frame}

\begin{frame}{Struktur}
\begin{enumerate}
  \item Beginne \textbf{nicht} mit einer Übersicht/Agenda/Gliederung/etc
  \item Ein einführendes Beispiel um das Problem zu erklären
  \item Beispielhaft den eigenen Lösungsansatz erklären
  \item Lösung im Allgemeinen erklären
  \item Evaluation und verwandte Arbeiten präsentieren
  \item Zusammenfassungsfolie als Abschluss
\end{enumerate}
\end{frame}

\begin{frame}{Inhalt}
Idealerweise ein durchgehendes Beispiel benutzen
mit dem man sowohl das Problem, die Lösung, und evtl. Randfälle erklären kann.
\vfill

Der Erfahrung nach braucht man 1-2 Minuten pro Folie.
Für einen 20 Minuten Vortrag sollten es also weniger als 20 Folien sein.
\vfill

Arbeitsaufwand knapp quantifizieren (Lines of Code, Anzahl Theoreme)
\vfill

Was nicht im Vortrag kommt geht unter!
Die Existenz nicht vortrags-fähiger Teile der Arbeit
(wie etwa länglicher Beweise) zumindest erwähnen.
\end{frame}

\begin{frame}{Vorkenntnisse}
Was man bei uns am Lehrstuhl voraussetzen kann:
\vspace{20pt}
\begin{itemize}
  \item libFirm/SSA Form auf einer Folie kurz einführen
  \item C, Java, Haskell, Lambda ist bekannt
  \item X10, ML, Agda, Scala kurz einführen
  \item Andere Programmiersprachen gut motivieren
\end{itemize}
\end{frame}

\begin{frame}{Vortrag}
Alleine Üben! Daheim vor dem Spiegel üben.
Einen Vortrag nur alleine, aber im Stehen und laut sprechend,
vorzutragen ist meist schon lehrreich im Vergleich
zu stillem Folienbasteln.
\vfill

Frei sprechen (oder zumindest die Illusion wahren)
\vfill

Laut und deutlich sprechen (leider häufiges Informatikerproblem)
\vfill

Live vorführen von implementierten, sofern gut geübt, macht Eindruck
\vfill

Zeitrahmen \textbf{nicht überziehen}!
Zwischenfragen gehen natürlich nicht vom Zeitkonto ab
\vfill

Die Folien dem Betreuer zur Durchsicht geben
\end{frame}

\begin{frame}{Bewertung}
\begin{itemize}
  \item Ca.\ 20\% der Gesamtnote kommen durch
    kommunikative soft-skill Aspekte zustande.
  \item Ein schlechter Vortrag kann leicht die Note um 0.7 nach unten ziehen.
\end{itemize}
\end{frame}

\begin{frame}{Links}
\begin{itemize}
  \item \href{http://beza1e1.tuxen.de/articles/technical_presentation.html}{9 Tips how to give a technical presentation}
  \item \href{http://andreas-zeller.blogspot.de/2013/10/summarizing-your-presentation-with.html}{Summarizing your presentation with miniature slides}
  \item \href{https://www.cs.cmu.edu/~kayvonf/misc/cleartalktips.pdf}{Tips for Giving Clear Talks}
\end{itemize}
\end{frame}

\end{document}
