\chapter{Design and Implementation}\label{sec:impl}

As outlined in section \ref{sec:relWork}, there have been attempts at using symbolic execution and model counting to measure how much information a program can leak. In many of these works, there have been attempts to mitigate the problems that are caused by path explosion in input programs and the computational limits of state of the art model counters. These attempts include using approximate model counting instead of a precise calculation (\cite{biondi18, chu19}) or transformations of SAT formulas into easier to handle representations (\cite{klebanov13}). We have not yet seen an attempt to use static (Q)IFC techniques to reduce complexity at the time of formula creation. \com{besser in related work? oder sogar intro?}

The analysis we will present in the following chapter combines static and dynamic approaches to the QIFC problem. 

We will run a static pre-processing to identify program parts that are critical to the flow of information and restrict the subsequent dynamic analysis to those parts.

The dynamic analysis finds and examines possible execution paths using symbolic execution and evaluates and information flow along these paths through an approximate model counter, similar to the techniques we have seen in \cite{klebanov13, biondi18, chu19}.
If during the analysis, the generated boolean predicates are still too complex to be evaluated by a model counter, out tool will split the program into segments and separately, either statically or dynamically, analyze each segment and combine the results for an overall estimation of the programs channel capacity.

The analysis is integrated in an interpreter that will execute the program for a given input and, additionally to the channel capacity, will give estimations for the size of the indistinguishability class of the given input.

\td{which guarantees does our analysis give?}

% input program:
% - javac --> bytecode
% working on CFG + SSA form

\section{Input Programs and Program Representation}\label{sec:inputLang}

Input programs are written in a variant of the \texttt{while}-language with Java-like syntax, that contains the following control structures, using their standard semantics:
\begin{itemize}
    \setlength\itemsep{0em}
    \item sequential composition
    \item assignments
    \item \texttt{if}-statements
    \item \texttt{while}-statements
    \item \texttt{break}-statements
    \item function calls (with return values)
\end{itemize}
All variables are signed integers of a fixed width $w$. The right hand side of an assignment is an expression that uses the standard arithmetic and bitwise boolean operators. Boolean expressions used in \texttt{while}- and \texttt{if}-statements are defined in the standard way.

We will denote secret inputs as $h_i$, constant values as $n_i$ and other program variables as $x_i$. The set of all input variables for a program will be denoted as \In and the set of all possible input sets is written as \allIn. A variable can be leaked to a public output channel via the special function \texttt{print}. We assume that all program executions terminate. More specifically, we assume that they terminate normally, without throwing any exceptions.

To analyze the input program \p, we use the program's functions' control flow graphs (CFG) in static single assignment (SSA) form.

\begin{itemize}
    \item \note{execution values: represent actual value during execution}
    \item \note{represented as bit vectors (two's complement}
    \begin{itemize}
        \item integers: width n
        \item booleans: width 1
    \end{itemize}
    \item \com{specify relationship between valueDeps and execVal}
\end{itemize}

\begin{definition}[Execution Value]
The function maps a program value to the number that was assigned during a particular execution. If in this execution, the value remains undefined, because the corresponding assignment instruction wasn't executed, the function will return $\bot$.
    \begin{center}
        $exec_{\mI, p}: \val_p \longrightarrow \{0, 1\}^n \cup \{\bot\}$
    \end{center}
The function is parameterized by the input \p and the set of program inputs \I, which determine the particular execution of \p that is evaluated.
\end{definition}

\begin{definition}[Value Dependencies]

    
\end{definition}

\section{Basic Analysis Design}

\begin{itemize}
    \item \note{approach: symbolic execution}
    \item \note{data + cf dependencies represented as boolean formulas}
    \item \com{define set of boolean formulas, \ttt, \fff, operators etc.}
\end{itemize}

We use propositional logic to track the way that information about the secret inputs flows through the program. Each program value is interpreted as a bit vector, where each bit is associated with a propositional formula that encodes how the state of the bit is dependent on the secret inputs. We will use subscript indices to access the i-th bit of a program value or the i-th element of a vector respectively. The bitwidth of a value \texttt{x} will be denoted as $width(\mathtt{x})$.

\begin{definition}[Independent Set]
    Let \texttt{H} be the set of input values of \pp.
    \begin{center}
        $\var_p := \bigcup\limits_{\mathtt{h} \in \mathtt{H}} \{h^j | 0 \leq j \leq width(\mathtt{h})\}$
    \end{center}
    The values of a program's input parameters are not dependent on any other value in the program. We define the set $\var_p$, that contains a propositional variable for each input bit. Every other value of \p is either constant or can be described by a propositional formula over $\var_p$.
\end{definition}

\textbf{\note{Implicit information flow (CF dependencies)}}

\begin{itemize}
    \item \note{Associate each block w/ a boolean formula that evaluates to true <==> block is executed in the specific execution}
    \item \note{a block is executed iff one of its predecessors is executed and if the jump comdition from that predecessors evaluates to true}
    \item \com{Function for accessing condition of a conditional kump at the end of block -->} \question{necessary?}
    \item \question{`Abbildungsvorschrift' nötig?}
    \item \com{imDep: bblock -> formula -- f === wahr <--> b wird ausgeführt}
\end{itemize}

% define implicit leakages
\paragraph{Implicit Information Flow}

% TODO: better description for the case conditions
% TODO: check if this is even correct lol
% TODO: is knowledge the right word to use here?
% TODO: find better name for function than Phi

\paragraph{Explicit Knowledge Function}

% map + semantics for finding implicit leakage

\section{Increasing Efficiency through Static Pre-Processing}
\begin{itemize}
    \item "Konstantenfaltung" auf Bitebene
    \item Backward-Slice des Ausgabewerts, nicht enthaltene Werte müssen nicht analysiert werden
    \item Backward-Slice von Konstanten Werten --> Werte, die die Ausgabe beeinflussen und Abhängigkeiten dadurch immer durch konstanten Wert fließen müssen nicht analysiert werden
\begin{itemize}
    \item kann mitberücksichtigt werden, wenn man vor dem "Ausgabe-Backward-Slice" alle Abhängigkeiten von konstanten Werten entfernt 
\end{itemize}
\end{itemize}

\section{Handling Loops, Recursion and Function calls}

\section{Combining static and dynamic analyses to estimate channel capacity}
\textbf{Goal:}
\begin{itemize}
    \item channel capacity c := number of distinct program outputs
    \item c large --> much leakage
    \item c small --> little leakage
\end{itemize}
Consider program w/ single loop. Precisely computing channel capacity might be infeasible due to too many possible execution paths through the loop

\begin{itemize}
    \item Define unrolling limit n
    \item Compute SAT formula $cond_n$ where $cond_n(h) == true$ iff input h leads to more than n loop iterations
    \item if $SAT(cond_n)$ we cannot analyze the loop in a purely dynamic manner
    \item \textbf{Switch to an analysis in parts}
\end{itemize}

\textbf{Analysis in parts:}
\begin{itemize}
    \item Let c = true channel capacity of the program
    \item divide program into 3 parts: $p_b$ - before loop, $p_l$ - loop body, $p_a$ - after loop
    \item compute channel capacities for all 3 parts separately: $k_b, k_l, k_a$
    
    \begin{itemize}
        \item take into account constant bits in result from previous program part for inputs for next part
        \item for loop part: only analyze a single loop iteration
        \item \com{Could be done with dynamic or static analysis}
    \end{itemize}
    
    \item each analysis (except the first) overapproximates the possible inputs for the program section --> computed channel capacities possibly larger than they actually are --> sound upper limit for leakage
    \item $c \leq k_b, k_l, k_a \implies c \leq \min{k_b, k_l, k_a}$
    \item \com{Correction: Implicitly assumes that loop is executed at least once}
    \begin{itemize}
        \item Use model counter to estimate number of inputs that would not enter the loop --> $k_n$
        \item worst case: these are all different to what the loop outputs are
        \item update $k_l += k_n$
        \item instead of $p_l$, estimate channel capacity of if (cond) $p_l$ else skip --> all inputs that don't enter the loop (produce outputs that are accounted for in $k_n$ will lead to the same output in $p_l$; avoid counting them double)
        \item \com{instead of $p_l$ directly, compute channel capacity of \\ \texttt{if (cond) res = loopBody else res = h return res}}
        \item set p := $loopBody^n$ if (loop cond) loopBody else return $\bot$ to find channel capacity for program runs that execute loop more than n times
        \item for executions that execute loop less than n times: count loop iterations as part of $p_b$
    \end{itemize}
    \item \com{all partial analyses could be done either statically or dynamically}
        \begin{itemize}
        \item \question{when to use what?}
        \item \question{shouldn't purely dynamic analysis be feasible for all program parts here since we `eliminated' the many loop iterations?}
        \item \note{might still not be possible bc loop might contain another loop, or recursive function, etc. --> further segmentation necessary}
        \item number of segments that need to be analysed separately explodes
        \item switch to static analysis when num. of segments gets too large, before then do everything dynamically to get more precise estimations
        \item \textbf{How to decide when to do what?}
        \begin{itemize}
            \item for each segment: keep track of `segmentation depth' $s$
            \item define segmentation limit $\hat{s}$
            \item if $s >= \hat{s}$, but segment can still not be feasibly analysed in a dynamic manner, switch to static 
        \end{itemize}
    \end{itemize}
    
    
    \item \question{How do we know, when to further partition the program?}
    \begin{itemize}
        \item number of loop iterations that possibly happen
        \item width of SAT formulas generated up until this point --> number of operands, number of nodes, number of literals
        
    \end{itemize}
     \item \com{\textbf{Alternative approach:}}
    \begin{itemize}
        \item use same method as for estimation of indistinguishability set
    \end{itemize}
    
\end{itemize}

\note{while (i < secret) { if (i > 64) {i = 100; break;} i++}}

\section{Integration with Interpreter}
\begin{itemize}
    \item channel capacity --> measures leakage in terms of the number of different outputs a program can produce
    \item even if there are only 2 different outputs, if one of those outputs can only be the result of one input, then the attacker will learn the whole secret
    \item interesting for user to know the number of inputs to the program that will produce the same output as the user's execution did --> `indistinguishability set' D
    \item D large --> leakage small
    \item D small --> leakage big
    \item precise computation of $|D|$ infeasible, we need to estimate
    \begin{itemize}
        \item Problem: loops + recursive calls produce SAT formulas that are too big to handle efficiently
    \end{itemize}
\end{itemize}

\textbf{Underapproximation of $|D|$ --> finding max amount of leakage for this run}\\
Consider program that contains a loop
\begin{itemize}
    \item not feasible to precisely analyse all possible numbers of loop iterations
    \item choose unrolling limit $n$
    \item get SAT formula $cond_n: cond_n == true$ iff input needs less than n iterations
    \item only seek candidates for $D$ among inputs that require less than $n$ loop iterations by adding $\land cond_n$ to Model Count formula --> $D_u$
    \item definitely $D_u \subseteq D$, so $|D_u|$ is sound upper bound for leakage
\end{itemize}
More precise approximation? So far we haven't considered executions ith $>n$ loop iterations at all
\begin{itemize}
    \item 
\end{itemize}

\textbf{Overapproximation of $D$ --> finding min amount of leakage for this run}
Consider program that contains a loop
\begin{itemize}
    \item not feasible to precisely analyse all possible numbers of loop iterations
    \item choose unrolling limit $n$
    \item get SAT formula $cond_n: cond_n == true$ iff input needs less than n iterations (same as above)
    \item formula that describes result of loop for $< n$ iterations: $b_{<}$
    \item introduce new vars: $v_>$
    \item set loop result as $cond_n ? b_{<} : v_>$ --> use that for model counting to find set $D_o$
    \item allows inputs with more than n iterations to produce arbitrary results from the loop
    \item every result that might actually occur is contained, but also one's that are not actually possible --> $D \subseteq D_o$, so $|D_o|$ is sound lower bound for leakage
\end{itemize}

\com{Possible to use a form of static analysis for estimation of indistinguishability set?}

\begin{itemize}
    \item static analyses determine the number of distinct outputs, not the size of the indistinguishability set
    \item \question{can we manipulate the program s.t. the number of distinct outputs is the same as the size of $D$?}
    \item desired output: $o$
    \item set $p_new(h)$ := res = p(h) if (res == o) return h else return $\bot$
    \item every input that is not part of $D$ gets mapped to the same output
    \item result will be overestimation of $|D|$ --> gives us \textbf{lower bound of leakage}
    \item can only work on the whole program, not on segments (would be a possible underapproximation of $D$ for middle segments) --> under- and overapproximating at the same time is not good...
    \item \com{underapproximation if we change (res == o) to (res != o) ??}
\end{itemize}

\section{Implementation}