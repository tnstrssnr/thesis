\chapter{Benchmarks}
The benchmarks follow the convention from before: secret inputs are called \In, while public outputs are called \Out. All public outputs are leaked at the end of the execution.

\section{Small Benchmarks}
The following set of benchmarks contains no loops or functions. The main purposes of the programs is to test the tools' ability to deal with simple operations and implicit flows.

\paragraph{Masked Copy}
This benchmark is taken from \cite{meng11}. The program masks the 16 highest value bits from the input and outputs the rest. The channel capacity is 16.

\begin{center}
    \begin{lstlisting}[language=C, caption=Masked Copy]
        int l = h & (-1 << 16);
    \end{lstlisting}
\end{center}


\paragraph{Sum Query}
This benchmark is taken from \cite{backes09}. The program adds together and then returns the three secret input values. The channel capacity is 32 bit.

\begin{center}
    \begin{lstlisting}[language=C, caption=Sum Query]
        int l = h1 + h2 + h3;
    \end{lstlisting}
\end{center}

\paragraph{Sanity Check}
The benchmark was taken from \cite{newsome09}. Because our tool is unable to represent unsigned integers, we added the condition $h >= 0$ to the conditional. We initialized the value \texttt{base} with 3. The channel capacity is 4.

\begin{center}
    \begin{lstlisting}[language=C, caption=Sanity Check]
        int l;
		if (0 <= h && h < 16) {
			l = 3 + h;
		} else {
			l = 3;
		}
    \end{lstlisting}
\end{center}

\paragraph{Table LookUp} This is a variant of benchmark found in \cite{newsome09}. If the input is within the right value range, a value from a pre-initialized array is returned. All other inputs return 0. The channel capacity is 3.

\begin{center}
    \begin{lstlisting}[language=C, caption=Table LookUp]
        int[] table;
		table[0] = 0;
		table[1] = 1;
		table[2] = 2;
		table[3] = 3;
		table[4] = 4;
		table[5] = 5;
		table[6] = 6;
		table[7] = 7;

		int l = 0;

		if (0 <= h && h < 8) {
			l = table[h];
		} else {
			l = 0;
		}
    \end{lstlisting}
\end{center}

\section{Loop Benchmarks}

\paragraph{Laundering Attack}
\begin{center}
    \begin{lstlisting}[language=C, caption=Laundering Attack]
        int out = 0;
		for (int i = 0; i < h; ++i) {
			++out;
		}
    \end{lstlisting}
\end{center}

\paragraph{Sum And Product}
The benchmark is the example program from figure \ref{fig:slice}. It iteratively computes the sum and the product of the two input values and leaks the sum. The program's channel capacity is 32.

\begin{center}
    \begin{lstlisting}[language=C, caption=Laundering Attack]
        int sum = h2;
		int product = 0;
		int i = 0;

		while (i != h1) {
			sum += 1;
			product += h2;
			i++;
		}
    \end{lstlisting}
\end{center}