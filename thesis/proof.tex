\chapter{Proof for Theorem \ref{eq:ev}}\label{ch:evProof}

We proof the equivalence theorem, that is introduced in section \ref{sec:design}:

\begin{theorem*}[Equivalence Theorem]
    Given an input value $h \in \mathcal{H}$ for the program $p$, let $v$ be an arbitrary value in the program $p$. The relation between the dependency vector and the execution value of v is given by:
    \begin{center}
        $\forall 0 \leq i < w: \mathcal{V}_h(dVec(v)^i) \iff \llbracket p \rrbracket_h (v)^i$
    \end{center}
\end{theorem*}

To prove the theorem, we need to prove the correctness of the definition of $\me{\cdot}$ (\ref{def:exprEval}) as well as the correctness of the definition of $exec(\cdot)$ (\ref{def:exec}).

\paragraph{Assumptions}
We assume that all expressions in the program $p$ contain at most one operator. The operands of the operator are constant values or variable / parameter accesses. Nested expressions are split: The inner expressions are assigned to new values which in turn are used as the operands for the outer expression.

\paragraph{Proof Structure}
The proof will be organised as three nested inductions. The outermost induction is a structural induction over the program p. The base case are linear programs, that contain no diverging control flow structures.

We will use an induction over the basic blocks of $p$ to prove the hypothesis holds for linear programs. For each step in the induction, we show that the theorem holds via an induction over the definition of $\me{\cdot}$ and a proof that the edge notations $follow(e)$ and the execution condition $exec(\cdot)$ are correct.

From the base case of linear programs, we will prove in the induction step, that the theorem holds for programs containing \texttt{if}-statements. This again will be done by an induction over the basic blocks of the program in question.

\section*{Structural Induction over the Program $p$}

\paragraph{Hypothesis}
Let $p$ be a loop-free program that consists of a single function. Then, for every basic block $b \in \mbb_p$, the following statements hold:
\begin{enumerate}
    \item[H.1] $\mathcal{V}_h(exec(b)) = \mttt \iff \text{b is executed in the execution with input $h$}$
    \item[H.2] For all values defined in $b$, theorem \ref{thm:equiv} is fulfilled
    \item[H.3] For all outgoing edges $e$ of $b$,\\ $\mathcal{V}_h(follow(e)) = \mttt \iff \text{The execution with input $h$ follows the edge $e$}$
\end{enumerate}

\subsection*{Base Case: Linear Programs}
Let $p$ be a linear program, i.e. a program without diverging control flow structures. Such a program will consist of a $\texttt{start}$ block, a block $b$ containing all program statements and an \texttt{end} block. $b$'s only predecessor is \texttt{start} and its only successor is \texttt{end}.

\subsubsection{Induction over $p$}
\paragraph{Base Case}
Let $b = \mathtt{start}$. Per definition, $exec(\mathtt{start}) = \mttt$. Since the start block is executed, independent of the input, H.1 holds.

The basic block \texttt{start} does not contain any statements, hence H.2 holds trivially.

\texttt{Start} has a single successor. For the edge $e$ that connects \texttt{start} to its successor, $follow(e) = \mttt$. H.3 holds, since every execution follows the edge $e$.

\paragraph{Induction Hypothesis}
Let $b \in \mbb_p$ be an arbitrary basic block of the program $p$. We assume that for all basic blocks that might precede $b$ in an execution, the statements H.1, H.2 and H.3 hold.

\paragraph{Induction Step}
We show that H.1, H.2 and H.3 hold for the block $b \in \mbb_p$, if the induction hypothesis is fulfilled.

Because $p$ is a linear program, $b$ has a single predecessor. Let $b'$ be this predecessor. At the same time, $b$ is the only successor of $b'$. Thus, $follow((b', b)) = \mttt$ and $exec(b) = exec(b') \land \mttt$. Since $exec(b')$ is correct due to the induction hypothesis and $b$ will be executed iff $b'$ is executed (because it is the only successor),  H.1 holds for the block $b$.

We prove the statement H.2 by induction over the definition of $\me{\cdot}$. Let $v \in \val_p$ be an arbitrary value in $p$, that is defined by the statement $v \leftarrow e$.

\subparagraph{Base Case}
\begin{itemize}
    \item $e := n, \quad n \in \mathbb{Z} \implies v_h = bv(n)$\\
    Per definition $\mathcal{E}(e) = bv(n)$. Thus $\forall 0 \leq i < w: \mathcal{V}_h(\mathcal{E}(e))^i \iff v_h^i$
    
    \item $e := \mIn \implies v_h = bv(h)$\\
    Per definition $\mathcal{E}(e) = \var(h)$ and $\mathcal{V}_h(\var(h)) = bv(h)$. Thus\\ $\forall 0 \leq i < w: \mathcal{V}_h(\mathcal{E}(e))^i \iff v_h^i$
    
\end{itemize}

\subparagraph{Induction Hypothesis}
    Let theorem \ref{thm:equiv} be true for every value $v'$ that appears in the expression $e$.
    
\subparagraph{Induction Step}
\begin{itemize}
    \item $e := v', \quad v' \in \val_p \implies v_h = \llbracket p \rrbracket_h (v')$\\
    For all inputs $h \in \mathcal{H}$, $\mathcal{V}_h(v) = \mathcal{V}_h(v') \stackrel{I.H.}{=} \llbracket p \rrbracket_h(v') \stackrel{v := v'}{=} \llbracket p \rrbracket_h(v)$
    \item $e := \hat{e} \oplus \tilde{e}$ / $ e := \oplus \hat{e}$ for a bitwise or arithmetic operator $\oplus$\\
    The function $\me{\cdot}$ is defined by the boolean algebra definitions of the operators. The correctness of the theorem \ref{thm:equiv} follows from the equivalence of the evaluation of boolean algebra and propositional logic.
\end{itemize}
Thus, theorem \ref{thm:equiv} holds for the value $v$ and H.2 is fulfilled for the block $b$.

Finally, H.3 is fulfilled for the basic block $b$, because $b$ has at most one outgoing edge $e$. If $b$ is the \texttt{end}-block, $b$ has no outgoing edges and H.3 is trivially fulfilled. If $b$ has one outgoing edge $e$, $follow(e) = \mttt$. H.3 holds, because very execution must follow $e$, since there are no other possibilities to continue the execution otherwise.\\[1em]

\paragraph{Conclusion}
For every linear program $p$, all basic blocks fulfill the statements H.1, H.2 and H.3.

\subsection*{Induction Hypothesis}
Due to the prerequisite of $p$ being loop-free and consisting of only a single function, the only control-flow structure we need to address in the induction are conditional statements. These can appear linearly in the program $p$ or nested inside one another.

Let $p$ be an arbitrary program that contains a conditional statement. Let $b_{split}$ be the basic block, where the conditional statements splits the control flow into two separate paths and let $b_{join}$ be the basic block where these two paths merge back together. The conditional statement does not contain any further conditional statements inside it.

Let $\mbb_{before} \subseteq \mbb_p$ be the set of blocks that might precede $\splitt$ in an execution of $p$, $\mbb_{during} \subseteq \mbb_p$ the set of basic blocks between $\splitt$ and $\join$ and $\mbb_{after} \subseteq \mbb_p$ the set of basic blocks that succeed $\join$ in an execution of $p$.

\paragraph{Hypothesis}
Let, for the program structure that be obtain if we remove $\mbb_{during}$ from the program $p$ and connect $\splitt$ and $\join$ by a single edge, the proof hypothesis be true.

\subsection*{Induction Step}
We show, that for all basic blocks in the program $p$, the statements H.1, H.2 and H.3 are true. We prove this by an induction over the basic blocks of $p$:

\subsubsection{Base Case}
The base case is the same as in the case of linear programs.

\subsubsection{Induction Hypothesis}
Let $b \in \mbb_p$ be an arbitrary basic block of the program $p$. We assume that for all basic blocks that might precede $b$ in an execution, the statements H.1, H.2 and H.3 hold.

\subsubsection{Induction Step}
We show that H.1, H.2 and H.3 hold for the block $b \in \mbb_p$, if the induction hypothesis is fulfilled.

We consider the following cases:
\begin{enumerate}
    \item $b \in \mbb_{before}$\\
    We know from the induction hypothesis, that we can build a program that contains the blocks $\mbb_{before}$ and that fulfills the proof hypothesis. Thus, for this program, the blocks in $\mbb_{before}$ all fulfill H.1, H.2 and H.3.
    Because for every basic block, the conditions H.1, H.2 and H.3 only depend on the block itself and its predecessors, they are still fulfilled in the program $p$.
    \item $b = \splitt$
    \begin{itemize}
        \item[H.1] If the block $\splitt$ is executed, then one of its predecessors $b'$ must have been executed before, followed by a transfer of the control flow via the edge $(b', b)$. Hence $exec(b') \land follow((b', b)) = \mttt$. Thus, by definition, $exec(b) = \mttt$. If on the other hand $exec(b):= \bigvee\limits_{b' \in pred(b)} (exec(b') \land follow((b', b))) = \mttt$, then there must a exist a predecessor $b'$ for which \\$exec(b') \land follow((b', b) = \mttt$. Per the induction hypothesis, this implies that $b'$ will be executed and the execution will follow the edge $(b', b)$. Thus $b$ will be executed as well.
        \item[H.2] That the condition H.2 holds for the block $\splitt$ can be proven with the same induction as in the base case.
        \item [H.3] Since the block $b$ splits the control flow, its outgoing edges are going to be annotated with the condition of the conditional jump. Let $e$ be the expression of the condition for the jump from $b$ to $b'$. If $e := \mttt$ or $e := \mfff$ then H.3 certainly holds. If $e := v_1 \simeq v_2$ for some comparison operator $\simeq$, then:
        $\me{e} := dVec(v_1) \simeq dVec(v_2)$. We have already shown that H.2 holds for $v_1$ and $v_2$. Due to the equivalence of comparison operators in binary and propositional logic, H.3 holds for $\splitt$.
    \end{itemize}
    
    \item $b \in \mbb_{during}$\\
    The branches of the conditional statement are linear program segments according to the prerequisites. Thus $b$ has one predecessor and one successor. We can prove that H.1, H.2 and H.3 hold, using the same arguments as in the base case for linear programs.
    
    \item $b = \join$\\
    \begin{itemize}
        \item[H.1] analog to $b = \splitt$
        \item[H.2] Let $v$ be an arbitrary value that is defined in $\join$ and let $v$ be defined by the expression $e$. We prove H.2 via an induction over the expression $e$. The induction is mostly analog to the induction done in the base case for linear programs. It is only left to prove that the condition H.2 holds, if $e := \phi(v_1, v_2)$.
        
        Let $e := \phi(v_1, v_2)$ and let $b_1, b_2$ be the first and second predecessor of $\join$. Then $dVec(v) := \mathbb{IF}(exec(b), v_1, v_2)$.
        
        Let $h \in \mathcal{H}$ be an input for which the block $\join$ is executed. Then one of the blocks $b_1, b_2$ must have been executed as well. Note that $exec(b_1) \not\iff exec(b_2)$. Let without loss of generality $exec(b_1) = \mttt$. Then:\\ $\mathcal{V}_h(dVec(v)) = \mathcal{V}_h(v_1) \stackrel{H.I}{\iff} \llbracket p \rrbracket_h (v_1) = \llbracket p \rrbracket_(v)$. 
        \item[H.3] analog to base case
    \end{itemize}
    \item $b \in \mbb_{after}$\\
    analog to case $b \in \mbb_{before}$
    \item otherwise\\
    If $b$ doesn't fall into one of the other cases, then it is a block inside a conditional statement, where the other branch of the conditional statement contains the blocks $\splitt, \mbb_{during},$ and $\join$. Since the two branches of a conditional statement cannot both be executed at the same time, their computations do not influence each other.
    Therefore we can prove the hypothesis holds analog to the cases $b \in \mbb_{before}$ and $b \in \mbb_{after}$.
\end{enumerate}