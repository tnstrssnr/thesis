\chapter{Proof for Equation \ref{eq:ev}}\label{ch:evProof}

Let $p$ be a deterministic program, with input \In and output \Out. Let $\mathcal{L}$ be the set of possible outputs.

\begin{align*}
    cc(p) &:= H_\infty(\mIn) - H_\infty(\mIn \: | \: \mOut) \\
    &= H_\infty(\mIn) - \sum\limits_{l \in \mathcal{L}} P[L = l]\: H_\infty(\mIn \: | \: \mOut = l) \\
    &= H_\infty(\mIn) - \sum\limits_{l \in \mathcal{L}} P[L = l]\: (-L_{dyn}(p, l) + H_\infty(\mIn)) \\
    &= H_\infty(\mIn) - \sum\limits_{l \in \mathcal{L}} P[L = l]\: H_\infty(\mIn) + \sum\limits_{l \in \mathcal{L}} P[L = l]\: L_{dyn}(p, l) \\
    &= \sum\limits_{l \in \mathcal{L}} P[L = l] \: L_{dyn}(p, l) \\
    &= \mathbb{E}(L_{dyn}(p,l))
\end{align*}

The equality in the second line results from the definition of $H( \mIn \: |\: \mOut)$, which is given in \cite{smith09}. All other steps in the proof use the definitions given in chapter \ref{ch:measures}.

\chapter{Proof of Theorem \ref{thm:equiv}}\label{ch:proofEquiv}

In this chapter we present a proof for the correctness of theorem \ref{thm:equiv}. We prove the correctness of the theorem for the programs addressed in section \ref{sec:design}, i.e. programs without loops, function calls and arrays.

The proof is presented in the following steps:
\begin{enumerate}
    \setlength\itemsep{0em}
    \item We show that theorem \ref{thm:equiv} holds for programs without diverging control flow
    \item proof correctness of edge annotations $follow(e)$
    \item proof correctness of $exec(b)$
    \item proof correctness of cf stuff
\end{enumerate}

\paragraph{Linear Programs}
Let $p$ be a program with linear control flow and let $v \in \val_p$ be an arbitrary value in $p$, that is defined by the statement $v \leftarrow e$.

Let $v_h := \llbracket p \rrbracket_h(v)$ be the bit vector of the execution value of $v$ for the execution with input an arbitrary but fixed input $h$. The dependency vector of $v$ is defined as $dVec(v) := \mathcal{E}(e)$. We show that $\forall 0 \leq i < w: \mathcal{V}_h(\mathcal{E}(e))^i \iff v_h^i$.

Distinction of cases for $e$:
\begin{enumerate}
    \item $e := n, \quad n \in \mathbb{Z} \implies v_h = bv(n)$\\
    Per definition $\mathcal{E}(e) = bv(n)$. Thus $\forall 0 \leq i < w: \mathcal{V}_h(\mathcal{E}(e))^i \iff v_h^i$
    
    
    \item $e := \mIn \implies v_h = bv(h)$\\
    Per definition $\mathcal{E}(e) = \var(h)$ and $\mathcal{V}_h(\var(h)) = bv(h)$. Thus\\ $\forall 0 \leq i < w: \mathcal{V}_h(\mathcal{E}(e))^i \iff v_h^i$
    
    
    \item $e := v', \quad v' \in \val_p \implies v_h = \llbracket p \rrbracket_h (v')$\\
    Assumption: Behauptung erfüllt für $v' \implies$ Behauptung gilt nach Voraussetzung
    
    \item 
    
\end{enumerate}
